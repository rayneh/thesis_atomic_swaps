\chapter{Related Work}
\label{ch:chapter03}
%theres also research specifically for ethereum private sidechains \cite{robinson2019atomic}
%from cross.chain swaps paper 
%CROSS CHAIN SWAPS ARE WELL KNOWN CONCEPT IN BLOCKCHAIN COMMUNITY [4,6,9,20,21,27] -> explain further in related work tho!!

The concept of two-party atomic cross-chain swaps is believed that it emerged from an online forum discussion, see footnotes for references to bitcoinwiki and bitcointalk threads \footfullcite{bitcoinwiki "Atomic cross-chain trading" - https://en.bitcoin.it/wiki/Atomic_swap. 2016} \footfullcite{T.Nolan. "Atomic swaps using cut and choose" - https://bitcointalk.org/index.php?topic=1364951. 2016}. There is open source code available on github \footfullcite{S. Bowe and D. Hopwood. "Hashed time-locked contract transactions" - https://github.com/bitcoin/bips/blob/master/bip-0342.mediawiki. 2020} \footfullcite{DeCred. "Decred cross-chain atomic swapping." - https://github.com/decred/atomicswap. 2020} and also Komodo \cite{barterdex2020} is releasing a two-party cross-chain swaps protocol for selected currencies, as well as another proposal for applications using swaps \cite{zyskind2018enigma}. To address the scalability limits of existing blockchains several off-chain payment networks do exist, for further details refer to \footfullcite{R. Network. "What is the raiden network?" - https://raiden.network/101.html. 2020} and to \cite{poon2016bitcoin} \cite{decker2015fast} \cite{green2017bolt}. These concepts process multiple transactions off-chain and eventually resolve the final balance through a single on-chain transaction. The Revive network \cite{khalil2017revive} is another actor which uses off-chain transactions in a way to ensure, that participating parties do not end off worse in the process of re-balancing. These algorithms also utilitize the concept of hashed timelock contracts, but they address a different set of problems. Multi-party swaps in example become important, when matching kidney donors and recipients. Think of a case, where a transplant recipient has an incompatible donor and wants to swap donor with another recipient where all parties want to assure, that each recipient obtains a compatible organ. A number of algorithms \cite{abraham2007clearing} \cite{dickerson2016position} \cite{jia2017efficient} have been proposed for matching donors and recipients. Shapley and Scarf \cite{shapley1974cores} write about the circumstances under which a certain kind of swap markets could have strong equilibrium. Kaplan \cite{kaplan2011improved} describes a polynominal-time algorithm, which constructs a swap digraph for a given set of swaps, if one exists. This paper focuses on "the clearing problem" though, which is roughly analogous to constructing a swap digraph, but not on how execute those atomic cross-chain swaps on blockchains. A precursor of atomic cross-chain swaps is the fair exchange problem \cite{franklin1998secure} \cite{micali2003simple}.

% explain fair exchange problem !!

%TODO add more recent related work!!  -> DO RESEARCH! (MB bring table of different swap technologies here -> see future research and outlook)
% ?? -> DeXTT: Deterministic Cross-Blockchain Token Transfers \cite{borkowski2019dextt}
% read paper -> Towards Blockchain Interoperability
% https://github.com/SuperNETorg/komodo/wiki/barterDEX-Whitepaper-v2  KOMODO WHITEPAPER
% MB interesting https://arxiv.org/ftp/arxiv/papers/1902/1902.04471.pdf

\chapter{Requirement Analysis and Concept}
\label{ch:chapter04}
WRITE ABOUT REQUIREMENT ANALYSIS AND CONCEPT HERE

\chapter{Implementation}
\label{ch:chapter05}
WRITE ABOUT IMPLEMENTATION HERE

\section{Reference Architecture}
\label{sec:chapter05:ref_architecture}

SHOW FIGURE FOR REFERENCE ARCHITECTURE HERE
%
% Section: Listen
%
\section{Smart Contracts}
\label{sec:chapter05:smartcontracts}
WRITE ABOUT SMART CONTRACT IMPLEMENTATIONS HERE

\section{Middleware (Intermediate Party)}
\label{sec:chapter05:middleware}
WRITE ABOUT MIDDLEWARE HERE

\section{Proof of Concept (aka Tests?)}
\label{sec:chapter05:poc}
EXPLAIN TESTING AND PROOF OF CONCEPT HERE

\chapter{Future Research and Outlook}
\label{ch:chapter06}
WRITE ABOUT FUTURE RESEARCH AND OUTLOOK HERE
non interactive proofs of poW \cite{kiayias2017non} sidechains \cite{kiayias2019proof} polkadot \cite{wood2016polkadot}  chain interoperability \cite{buterin2016chain} sharding \cite{buterin2017sharding} plasma \cite{poon2017plasma} z-snarks \cite{garoffolo2020zendoo} towards secure interoperability with smart contracts \cite{dagher2017towards}

caught in chains mention open tasks from paper \cite{borkowski2018caught}
pegged sidechains \cite{back2014enabling}
sidechains and interoperability \cite{johnson2019sidechains}



COSMOS \cite{kwon2018network} and tendermint \cite{kwon2014tendermint} buchmann \cite{buchman2016tendermint} PBFT \cite{castro1999practical}

pick some current research and make MB different chapters

%FROM \cite{li2019research}
Table 1. Cross-chain technology comparison
Cross-chain technology Algorithmic principle Shortcoming
Notary Transfer of assets to notary accounts Weak centralization
Side chain Read the data in the side chain Verification difficulties
Relay chain Establishing relay chain Data redundancy

Notary or multi-signature scheme is a weak centralized solution, transfer digital
assets by locking assets in the main chain to multiple specific addresses and requiring
signatures from multiple notaries. Ripple of Interledger protocol [2] enables assets of
different blockchains to be transferred across chains through trusted third parties. Notary
or multi-signature scheme is the simplest way to realize cross-chain asset transfer, but
the security of asset transfer depends on the honesty of notaries.
Side chain technology transfers digital assets by reading the data in the main chain
to verify the authenticity of transaction payments. BTC Relay project is to realize crosschain
payment by verifying Bitcoin SPV payment path to trigger the execution of ETH
smart contract [3].
Relay technology is to link existing blockchain projects by building a relay chain to
realize asset transfer of different blockchains. COSMOS [4] and Polkadot [5] projects
are to build cross-chain platform of blockchain through the combination of notary and
relay technology.

\chapter{Conclusion}
\label{ch:chapter07}
WRITE THE CONCLUSION HERE

\chapter{Terminology}
\label{ch:chapter08}
EXPLAIN THE TERMINOLOGY HERE

Address
Alt-Coin
App
Asset
Atomic Transaction
Autonomous Object
Block
Block Time
Bytecode
Cryptocurrency
CryptoNote
Contract 
External Actor
Ether
Ethereum Virtual Machine
Gas
Genesis Block
Hash
Padding - what is padding in sha-256 context
Protocol - what is a protocol inctroducing ethereum and solidity paper
Public and Private Key
Public and Private Blockchain
Storage State
Solidity
Transaction 
Token
Nonce
Miner

%on-chain / off-chain







