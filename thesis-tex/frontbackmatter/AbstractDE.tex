%*******************************************************
% Abstract in German
%*******************************************************
\begin{otherlanguage}{ngerman}
	\pdfbookmark[0]{Zusammenfassung}{Zusammenfassung}
	\chapter*{Zusammenfassung}
	Das heutige Blockchain-Ökosystem hat sich durch die Implementation verschiedener Konsensusalgorithmen und Architekturen in viele verschiedene Blockchains aufgeteilt. Nach einem Jahrzehnt der Blockchain-Anwendungen und -Entwicklungen besteht ein Bedarf für Interoperabilität zwischen kommenden und aktuellen Blockchain-Innovationen. Die Interoperabilität stellt aufgrund der architektonischen Unterschiede der Blockchain-Technologien eine komplexe Aufgabe dar, um somit Assets und Token in einer stetig wachsenden Umgebung zwischen Blockchains auszutauschen. Dieses Ökosystem besteht aus vielen Blockchains ohne Interoperabilität. Wenn Alice beispielsweise ihre Bitcoin mit Bobs Ether tauschen möchte, müssen sie zentralisierte Börsen für Kryptowährungen verwenden und sind da durch der Möglichkeit von Betrug und böswilligen Hacks ausgesetzt. Diese Arbeit liefert einen Prototyp mit einem Drei-Wege-Austausch zwischen drei Ethereum-Blockchains, um einen solchen Austausch zwischen Bitcoin und Ethereum durchzuführen sind weitere Untersuchungen erforderlich, da die Skriptsprache von Bitcoin derzeit nicht für den Austausch geeignet ist. Atomic Cross-Chain Swaps schlagen einen kleinen Schritt in Richtung Interoperabilität von Blockchains vor, indem diese Zwischenparteien wie Börsen entfernen. Die Blockchain-Interoperabilität bietet mehr Sicherheit für den Endbenutzer und schließt Angriffsmethoden für betrügerische Akteure.
\end{otherlanguage}
